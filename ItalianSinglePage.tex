\documentclass{tccv}
\usepackage[italian]{babel}
\usepackage[utf8]{inputenc}
\usepackage[T1]{fontenc}
\begin{document}

\part{Davide Macrelli}

\begin{eventlist}

\item{Maggio 2018- Oggi}
     {Alexide Srl, Cesena}
     {Programmatore Senior}

Sviluppo e test di applicativo web documentale con ASP.NET e database relazionale. Sviluppo e manutenzione di applicativo web per visualizzazione grafica usando libreria Three.Js e WebVR su stack Node.Js e Webpack. Progettazione e sviluppo di  un sistema per l'estrazione e analisi dati da repository svn, usando Python e libreria pandas. Progettazione e sviluppo di configuratore grafico, usando Windows forms e Python.

Sviluppo e manutenzione di applicativi Windows Forms gestionali. Partecipazione a tutte le fasi di realizzazione progetto software. Assistenza diretta con i clienti.

\noindent\hfil\rule{0.3\textwidth}{.4pt}

\item{Maggio 2017- Maggio 2018}
     {Alexide Srl, Cesena}
     {Programmatore Junior}

Sviluppo e test di applicativo web documentale con ASP.NET e database relazionale. Manutenzione di applicativi Windows Form gestionali. 
Assistenza diretta con i clienti.

\noindent\hfil\rule{0.3\textwidth}{.4pt}

\item{Settembre 2016- Dicembre 2016}
     {Alma Mater Studiorum, Cesena}
     {Tirocinante}

Progettazione e sviluppo di un applicativo web per la compilazione di checklist chirurgiche. L'obbiettivo era quello di basarsi su tecnologie e framework come Angular, per sviluppare un interfaccia intuitiva e flessibile per la compilazione delle checklist chirurgiche già ampiamente usate negli ospedali.

\noindent\hfil\rule{0.3\textwidth}{.4pt}

\item{Marzo 2015- Maggio 2015}
     {Alma Mater Studiorum, Cesena}
     {Tirocinante}

Progettazione e sviluppo di un sistema di streaming video tra device smartglass e un applicativo web. L'obbiettivo era di permettere una comunicazione bidirezionale tra il dispositivo a realtà aumentata e l'applicativo web. Le principali tecnologie utilizzate sono Android SDK e Java.



\end{eventlist}

\personal
    [https://github.com/DavideMacrelli]
    {Piazzale Fratelli Ruffini, 10, Cesena}
    {mcdavide92@gmail.com}
    {3297473001}
    


\section{Formazione}

\begin{yearlist}

\item[Laurea triennale]{2011-2017}
{ingegneria elettronica, informatica e telecomunicazioni}
{Università di Bologna, Campus Cesena}

\item[Diploma Scuola secondaria]{2006-2011}
{Tecnico industriale}
{Istituto Tecnico Marie Curie, Savignano}



\end{yearlist}

\section{Progetti pubblici}

\begin{yearlist}

\item{2017}
     {surgical-checklist (\href{https://github.com/DavideMacrelli/surgicalchecklist}{https://github.com/DavideMacrelli/s-rgicalchecklist})}
     {Web application per checklist chirurgiche}


\end{yearlist}

\section{Competenze tecniche}

\begin{factlist}

\item{Tecniche}
     {Python,  C\#, JavaScript, Node.js, Webpack, typescript, Angular, html, css, Sass, ASP.NET, MySQL, Java, pandas, threeJS, WebVR, git, svn, JSON,}


\item{Sto studiando}
     {React, Docker}


\end{factlist}

\end{document}